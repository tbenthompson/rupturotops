%        File: proof.tex
%     Created: Tue Dec 17 04:00 PM 2013 E
% Last Change: Tue Dec 17 04:00 PM 2013 E
%     Handy hints for vim-latex-suite: http://vim-latex.sourceforge.net/documentation/latex-suite/auc-tex-mappings.html
%     Character list for math: http://www.artofproblemsolving.com/Wiki/index.php/LaTeX:Symbols
%
\documentclass[a4paper]{article}

\usepackage[pdftex,pagebackref,letterpaper=true,colorlinks=true,pdfpagemode=none
,urlcolor=blue,linkcolor=blue,citecolor=blue,pdfstartview=FitH]{hyperref}

\usepackage{amsmath,amsfonts}
\usepackage{graphicx}
\usepackage{color}

\setlength{\oddsidemargin}{0pt}
\setlength{\evensidemargin}{0pt}
\setlength{\textwidth}{6.0in}
\setlength{\topmargin}{0in}
\setlength{\textheight}{8.5in}

\setlength{\parindent}{0in}
\setlength{\parskip}{5px}

\begin{document}
This is pretty loose and maybe I should improve it\dots I assume the proper choice of norm without specifying what it is. 
But, I think it gets the idea across. I use $O(\Delta t)$ notation very freely\dots I'm certain that there are some
errors, but I don't think that they influence the final result.

I have the following system of differential equations:
\begin{equation}
    \dot{\tau} = \nabla v - \frac{\mu}{\eta}\tau
    \label{eq1}
\end{equation}
\begin{equation}
    \nabla \cdot \tau = 0
    \label{eq2}
\end{equation}
where $\tau$ is the (vector) shear stress and $v$ is the (scalar) velocity in an
antiplane strain scenario.

I expand the velocity, $v$ in a Taylor series around the initial velocity, $v_n$.
\begin{equation}
    v = v_n + \dot{v}\Delta t + O(\Delta t^2)
    \label{taylorv}
\end{equation}
I insert this into Equation \ref{eq1}.
\begin{equation}
    \dot{\tau} = \nabla (v_n + \dot{v}\Delta t + O(\Delta t^2)) - \frac{\mu}{\eta}\tau
    \label{eq1new}
\end{equation}

Now, I perform the splitting.
The first equation is an initial value ODE to find the value of the tentative stress update, $\hat{\tau}(t_n + \Delta t)$. 
I ignore velocity evolution in this step.
\begin{equation}
    \dot{\hat{\tau}} = \nabla (v_n + O(\Delta t)) - \frac{\mu}{\eta}\hat{\tau}
    \label{split1}
\end{equation}
with $\hat{\tau}(t <= t_n) = \tau(t <= t_n)$ (they match exactly for all prior time).
Using a second order Backward Differentiation Formula (BDF2) method for the ODE, I can write
a discrete version of the time derivative:
\begin{equation}
    \frac{1}{2\Delta t}(3\hat{\tau}_{n+1} - 4\hat{\tau}_n + \hat{\tau}_{n-1}) = \nabla (v_n + O(\Delta t)) - \frac{\mu}{\eta}\hat{\tau}_{n+1}
    \label{split1}
\end{equation}
Rearranging the equation gives:
\begin{equation}
    \hat{\tau}_{n+1} = \frac{2\Delta t}{3}(4\hat{\tau}_n - \hat{\tau}_{n-1} + \nabla v_n - \frac{\mu}{\eta}\hat{\tau}_{n+1}) + O(\Delta t^2)
    \label{split1}
\end{equation}
From the derivation of the BDF2 formula, I know that:
\begin{equation}
    \|\hat{\tau}_{n+1} - \hat\tau(t = t_n + \Delta t)\| \leq C_1\Delta t^2
    \label{BDF2error}
\end{equation}

I've carried through the error term that resulted from dropping velocity evolution, so that:
\begin{equation}
    \|\hat\tau(t = t_n + \Delta t) - \tau(t = t_n + \Delta t)\| \leq C_2\Delta t^2
    \label{velocityerror}
\end{equation}

The total error in the ODE step is:
\begin{equation}
    \|\hat\tau_{n+1} - \tau(t = t_n + \Delta t)\| \leq \|\hat\tau(t = t_n + \Delta t) - \tau(t = t_n + \Delta t)\| + \|\hat{\tau}_{n+1} - \hat\tau(t = t_n + \Delta t)\|
    \label{velocityerror}
\end{equation}
\begin{equation}
    \|\hat\tau_{n+1} - \tau(t = t_n + \Delta t)\| \leq C_3\Delta t^2
    \label{totalOdeError}
\end{equation}

Returning to the description of the second step as a projection onto a divergence free function space, this
error bound confirms my intuition that the tentative stress update won't be very far from being divergence free,
and thus the projection step is a small correction. In fact, as long as the projection step doesn't make the error in the stress 
\textit{worse}, it's true ``purpose'' is to get a good updated velocity estimate.

The second step is:
\begin{equation}
    \frac{3}{2\Delta t}(\tau_{n+1} - \hat\tau_{n+1}) = \mu\nabla\left(\frac{\partial v}{\partial t}\Delta t\right) = \mu\nabla(v(t = t_n + \Delta t) - v_n + O(\Delta t^2))
    \label{step2}
\end{equation}
To complete the error analysis, I need to analyze the Poisson update to the velocity. 
This will give the error between the estimated new velocity, $v_{n+1}$, and the
true new velocity, $v(t = t_n + \Delta t)$.

In exact form, this is:
\begin{equation}
    -\nabla^2(v_{n+1} - v_n + O(\Delta t^2)) = \frac{3}{2\mu\Delta t}(\hat\tau(t = t_n + \Delta t)) = \frac{3}{2\mu\Delta t}(\hat\tau_{n+1} + O(\Delta t^2))
    \label{exact_poisson}
\end{equation}
Moving both the $O(\Delta t^2)$ terms to the RHS, the error can be viewed as a difference in forcing.

Using a finite element discretization makes the weakest assumptions about smoothness. 
The Lax-Milgram Theorem holds for the Poisson equation in weak form so that I have a continuous
dependence of the gradient of the solution on the forcing.
\begin{equation}
    \|\nabla (v_{n+1} - v_{n})\| \leq C_4\|\nabla \cdot \hat\tau_{n+1}\|
    \label{laxmilgram}
\end{equation}
And so looking at the error:
\begin{equation}
    \|\nabla (v_{n+1} - v(t = t_n + \Delta t))\| \leq C_5\Delta t^2 \implies \|v_{n+1} - v(t = t_n + \Delta t)\| \leq C_6\Delta t^2
    \label{laxmilgram error}
\end{equation}
Using a finite difference formulation gives a similar bound, but requires more stringent
smoothness assumptions (I think $C^4$ rather than $C^1$).

This closes our string of errors showing that all the errors are $O(\Delta t^2)$. 

\begin{equation}
    \|\tau(t = t_n + \Delta t) - \tau_{n+1}\| \leq C_7\Delta t^2
\end{equation}




\end{document}


